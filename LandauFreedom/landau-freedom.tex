\documentclass{article}

\usepackage{mathtools}
\usepackage{amssymb}
\usepackage{fullpage}
\usepackage{datetime}

% New definition of square root:
% it renames \sqrt as \oldsqrt
\let\oldsqrt\sqrt
% it defines the new \sqrt in terms of the old one
\def\sqrt{\mathpalette\DHLhksqrt}
\def\DHLhksqrt#1#2{%
\setbox0=\hbox{$#1\oldsqrt{#2\,}$}\dimen0=\ht0
\advance\dimen0-0.2\ht0
\setbox2=\hbox{\vrule height\ht0 depth -\dimen0}%
{\box0\lower0.4pt\box2}}

\newcommand\defeq{\stackrel{\mathclap{\normalfont\mbox{\tiny def}}}{=}}
\newcommand{\logg}[1]{\log \left\{ #1 \right\}}
\newcommand{\Ree}[1]{\Re \left\{ #1 \right\}}
\newcommand{\Imm}[1]{\Im \left\{ #1 \right\}}
\newcommand{\basis}[1]{\mathbf{ \hat{ #1 }}}

\renewcommand{\div}[1]{\nabla \cdot \mathbf{#1}}


\title{Locking the Wavefunction Freedom to Continuity for Landau Problem}
\author{Drew Silcock}
\date{}

\begin{document}

\maketitle

\section{Introduction}

\section{Analyticity of $f(x,y)$}

\textbf{Todo:} Find out why $f(x,y)$ has to be analytic

As $f(x,y)$ is analytic, this implies that it is harmonic and thus obeys the
Laplace equation:
\begin{equation}
    \nabla^2 f(x,y) = \partial_x^2 f(x,y) + \partial_y^2 f(x,y) = 0
\end{equation}
As $f(x,y)$ is independent of $z$ so $\partial_z f(x,y)$ is identically null.
This implies that the real and imaginary parts of $f(x,y)$, $f_R = \Re{\left\{ f
\right\}}$ and $f_I = \Im{\left\{ f \right\}}$, also obeys the Laplace equation:
\begin{align}
    \nabla^2 f_R &= 0 \\
    \textrm{and} ~ \nabla^2 f_I &= 0
\end{align}
In addition to this, $f(x,y) ~ \textrm{is analytic} \Leftrightarrow f(x,y)$
obeys Cauchy-Riemann (CR) equations:
\begin{align}
    \partial_x f_R &= \partial_y f_I \\
    \textrm{and} ~ \partial_y f_R &= - \partial_x f_I
\end{align}
These will come in useful shortly when examining $\div{J}$.

\section{Analyticity of $\logg{f(x,y)}$}

If we ignore the vertex points at $f(x,y) = 0$, then we can say that
$\logg{f(x,y)}$ is also harmonic and obeys the Laplace equation. Thus, writing
$f(x,y)$ as $f(x,y) = Fe^{i\phi}$ where $F, \phi \in \mathbb{R}:

\begin{align}
    \nabla^2 \logg{f(x,y)} &= \nabla^2 

\section{Locking the analyticity of $f(x,y)$ to the continuity of $\mathbf{J}$}

\end{document}
