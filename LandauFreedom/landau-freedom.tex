\documentclass{article}

\usepackage{mathtools}
\usepackage{amssymb}
\usepackage{graphicx}
\usepackage[super]{nth}
\usepackage{subcaption}
\usepackage{cleveref}

\DeclareGraphicsExtensions{.pdf,.eps,.png,.jpg}
\graphicspath{{figures/}}

% New definition of square root:
% it renames \sqrt as \oldsqrt
\let\oldsqrt\sqrt
% it defines the new \sqrt in terms of the old one
\def\sqrt{\mathpalette\DHLhksqrt}
\def\DHLhksqrt#1#2{%
\setbox0=\hbox{$#1\oldsqrt{#2\,}$}\dimen0=\ht0
\advance\dimen0-0.2\ht0
\setbox2=\hbox{\vrule height\ht0 depth -\dimen0}%
{\box0\lower0.4pt\box2}}

\newcommand{\defeq}{\stackrel{\mathclap{\normalfont\mbox{\tiny def}}}{=}}
\newcommand{\dbydt}[1]{\frac{\partial #1}{\partial t}}
\newcommand{\logg}[1]{\log \left\{ #1 \right\}}
\newcommand{\Ree}[1]{\Re \left\{ #1 \right\}}
\newcommand{\Imm}[1]{\Im \left\{ #1 \right\}}
\newcommand{\basis}[1]{\mathbf{\hat #1 }}
\newcommand{\mymu}[0]{\frac{m \Omega}{\hbar}}
\newcommand{\colvec}[3]{\begin{pmatrix} #1 \\ #2 \\ #3 \end{pmatrix}}

\renewcommand{\div}[1]{\nabla \cdot \mathbf{ #1 }}


\title{Locking the Wavefunction Freedom to Continuity for Landau Problem}
\author{Drew Silcock}
\longdate{}

\begin{document}

\maketitle

\section{Introduction}

We previously saw the ground state wavefunction solutions to the Landau problem
-- a constant-everywhere magnetic field $\mathbf{B} = B\basis{z}$ -- for the
straight and circular gauges.

This article aims to investigate the requirement that the function factor
$f(x,y)$ in $\Psi_{c,0}(x,y,t)$ be analytic with the requirement of continuity,
namely $\div{J} + \partial_t \rho = 0$.

\section{Circular gauge wavefunction and probability current for Landau problem}

When choosing the circular gauge, $\mathbf{A} = \frac{1}{2} B \left( x\basis{y}
- y\basis{x} \right)$, the ground state wavefunction solution is:
\begin{align}
    \Psi_{c,0}(x,y,t) = \mathcal{N} f(x,y)
    e^{- \frac{1}{2} \mu \left( x^2 + y^2 \right)} e^{-i \frac{1}{2} \hbar
    \Omega t}
\end{align}
Where $\mathcal{N}$ is a generic normalisation factor, $\mu = \frac{1}{2} \frac{M
\Omega}{\hbar}$ is written for brevity, $M$ is the mass, $\Omega$ is the angular
frequency of the wavefunction and $f(x,y)$ is an analytic function representing
the degeneracy present in this two-dimensional Gaussian hill, originating from
the freedom in choosing the original gauge. If we write $f(x,y)$ as $f(x,y) =
F e^{i \phi}$, where $F, \phi \in \mathbb{R}$, then the probability current is:
\begin{align}
    \mathbf{J}_c = F^2 \mathcal{M} e^{- \mu \left(
        x^2 + y^2 \right)} \left( \nabla \phi - \mu \left( x \basis{y}
        - y \basis{x} \right) \right)
\end{align}
Where $\mathcal{M} = \frac{\hbar}{M} \mathcal{N}^2$ is a constant of
normalisation, contracted for brevity.

\section{Analyticity of $f(x,y)$}

\textbf{Todo:} Find out why $f(x,y)$ has to be analytic

As $f(x,y)$ is analytic, this implies that its real and imaginary parts are
harmonic and thus obey the Laplace equation:
\begin{align}
                   \nabla^2 f_R &= \nabla^2 \Ree{f(x,y)} = 0 \\
    \textrm{and} ~ \nabla^2 f_I &= \nabla^2 \Imm{f(x,y)} = 0 \\
    \Rightarrow    \nabla^2 f   &= \nabla^2 f_R + \nabla^2 f_I = 0
\end{align}
In addition to this, $f(x,y) ~ \textrm{is analytic} \Leftrightarrow f(x,y)$
obeys the Cauchy-Riemann (CR) equations:
\begin{align}
    \partial_x f_R &= \partial_y f_I \\
    \textrm{and} ~ \partial_y f_R &= - \partial_x f_I
\end{align}
These will come in useful shortly when examining $\div{J}$.

\section{Analyticity of $\logg{f(x,y)}$}

For any function $g(z)$, if $g(z)$ is analytic on $\mathbb{C}$, then
$\logg{g(z)}$ is analytic on $\mathbb{C} \setminus \left\{ g(z) \in
\mathbb{R} : g(z) \leq 0 \right\}$ % TODO: Find reference for this

Thus, if we ignore the negative real line of $f(x,y)$, $f(x,y) \in \mathbb{R} :
f(x,y) \leq 0$, and take the principal branch of $\log$, then we can say that
the real and imaginary parts of $\logg{f(x,y)}$ are also harmonic. Writing
$f(x,y)$ as $f(x,y) = F e^{i \phi}$ where $F, \phi \in \mathbb{R}$:
\begin{align}
    \logg{f(x,y)} &= \logg{F} + i \phi \\
    \Rightarrow \Ree{\logg{f(x,y)}} &= \logg{F} \\
    \textrm{and} ~ \Imm{\logg{f(x,y)}} &= \phi
\end{align}
Then we can write the condition of harmonicity of real and imaginary parts of
$\logg{f(x,y)}$ as:
\begin{align}
    \nabla^2 \Ree{\logg{f(x,y)}} &= \nabla^2 \logg{F} = 0 \\
    \textrm{and} ~ \nabla^2 \Imm{\logg{f(x,y)}} &= \nabla^2 \phi = 0
\end{align}
In addition, $\logg{f(x,y)}$ also obeys CR:
\begin{align}
    \partial_x \left( \logg{F} \right) &= \partial_y \phi \\
    \textrm{and}~ \partial_y \left( \logg{F} \right) &= - \partial_x \phi
\end{align}
These will be useful later when examining $\div{J}$ and imply an important
result for $\left(\nabla F\right) \cdot \left(\nabla \phi\right)$.

\section{Implication that $\left(\nabla F\right) \cdot \left( \nabla \phi \right) = 0$}

As $\nabla^2 f(x,y) = 0$, we can use the product rule to find an important
result for $\left(\nabla F\right) \cdot \left(\nabla \phi\right)$:
\begin{align}
    \nabla^2 f &= \nabla^2 \left( F e^{i \phi} \right) \\
               &= \nabla \cdot \nabla \left( F e^{i \phi} \right) \\
               &= \nabla \cdot \left[ e^{i \phi} \nabla F + F \nabla \left( e^{i \phi} \right)
                  \right] \\
               &= \nabla \left( e^{i \phi} \right) \cdot (\nabla F) + e^{i \phi}
                  \nabla^2 F + \left( \nabla F \right) \cdot \nabla \left( e^{i \phi} \right)
                  + F \nabla^2 \left( e^{i \phi} \right) \\
               &= 2 \left( \nabla F \right) \cdot \nabla \left( e^{i \phi}
                  \right) + e^{i \phi} \nabla^2 F + F \nabla^2 \left( e^{i \phi}
                  \right)
\end{align}
As $\nabla \left( e^{i \phi} \right) = i e^{i \phi} \nabla \phi$, we can write:
\begin{align}
    \nabla^2 \left( e^{i \phi} \right) &= \nabla \cdot \left(i e^{i \phi} \nabla
        \phi \right) \\
    &= i \nabla \left( e^{i \phi} \right) \cdot (\nabla \phi) + i e^{i \phi}
        \nabla^2 \phi \\
    &= i \nabla \left( e^{i \phi} \right) \cdot (\nabla \phi) ~\textrm{as}~
        \nabla^2 \phi = 0 \\
    &= - e^{i \phi} \left( \nabla \phi \right)^2
\end{align}
Then $\nabla^2 f$ can be simplified to:
\begin{align}
    \nabla^2 f &= 2i e^{i \phi} \left( \nabla F \right) \cdot \left( \nabla \phi
        \right) + e^{i \phi} \nabla^2 F - F e^{i \phi} \left( \nabla \phi \right)^2 \\
    &= e^{i \phi} \left[ 2i (\nabla F) \cdot (\nabla \phi) + \nabla^2 F - F
        \left( \nabla \phi \right)^2 \right]
\end{align}
This can be simplified using the fact that $\nabla \left( \logg{F} \right) =
\frac{\nabla F}{F}$ and thus $\nabla F = F \nabla \left( \logg{F} \right)$:
\begin{align}
    \nabla^2 F &= \nabla \cdot \nabla F \\
               &= \nabla \cdot \left[ F \nabla \left( \logg{F} \right) \right]
    \\
    &= \left( \nabla F \right) \cdot \nabla \left( \logg{F} \right) + F \nabla^2
    \left( \logg{F} \right) \\
    &= \left( \nabla F \right) \cdot \nabla \left( \logg{F} \right)
    ~\textrm{as}~ \nabla^2 \left( \logg{F} \right) = 0 \\
    &= \left( F \nabla \left[ \logg{F} \right] \right) \cdot \nabla \left(
    \logg{F} \right) \\
    &= F \left[ \nabla \left( \logg{F} \right) \right]^2
\end{align}
Thus:
\begin{align}
    \nabla^2 F - F \left( \nabla \phi \right)^2 &= F \left[ \nabla \left( \logg{F}
        \right) \right]^2 - F \left( \nabla \phi \right)^2 \\
    &= F \left\{ \left[ \partial_x \left( \logg{F} \right) \right]^2
       + \left[ \partial_y \left( \logg{F} \right) \right]^2
       - \left( \partial_x \phi \right)^2
       - \left( \partial_y \phi \right)^2 \right\} \\
    &= F \left\{ \left( \partial_y \phi \right)^2
       + \left( - \partial_x \phi \right)^2
       - \left( \partial_x \phi \right)^2
       - \left( \partial_y \phi \right)^2 \right\} ~\textrm{using CR}\\
    &= 0
\end{align}
Then, $\nabla^2 f(x,y)$ is simply:
\begin{equation}
    \nabla^2 f(x,y) = 2 i e^{i \phi} \left( \nabla F \right) \cdot \left( \nabla
        \phi \right)
\end{equation}
As $\nabla^2 f(x,y) = 0$ and $\forall \phi \left( e^{i \phi} \neq 0 \right)$,
this gives the important result:
\begin{equation}
    \left( \nabla F \right) \cdot \left( \nabla \phi \right) = 0
\end{equation}


\section{Locking the analyticity of $f(x,y)$ to the continuity of $\mathbf{J}$}

Now we can examine what these facts about $f(x,y)$ say about the quantity
$\div{J}$:
\begin{align}
    \div{J}_c &= \nabla \cdot \left[ F^2 \mathcal{M}^2 e^{-
        \mu \left( x^2 + y^2 \right)} \left(
        \nabla \phi - \mu \left( x\basis{y} -
        y\basis{x} \right) \right) \right] \\
    &= \mathcal{M} \left\{ \left[
        \nabla \left( F^2 e^{- \mu \left( x^2 + y^2 \right)} \right) \right]
        \cdot \left[ \nabla \phi - \mu \left( x\basis{y} - y \basis{x} \right)
        \right]
        + F^2 e^{- \mu \left( x^2 + y^2 \right)} \left[ \nabla^2 \phi - \mu \nabla
        \cdot \left( x\basis{y} - y\basis{x} \right) \right]
        \right\}
\end{align}
But $\nabla^2 \phi = 0$ from earlier, as $\logg{f(x,y)}$ is analytic on
$\mathbb{C} \setminus \left\{ f(x,y) \in \mathbb{R} : f(x,y) \leq 0 \right\}$, and:
\begin{align}
    \nabla \cdot \left( x\basis{y} - y\basis{x} \right)
    &= \partial_y \left( x \right) - \partial_x \left( y \right) \\
    &= 0
\end{align}
So the divergence of $\mathbf{J}_c$ can be simplified to:
\begin{align}
    \div{J}_c = \mathcal{M} \left\{
        \left[ \nabla \left( F^2 e^{- \mu \left( x^2 + y^2 \right)} \right)
            \right] \cdot
            \left[ \nabla \phi - \mu \left( x \basis{y} - y \basis{x} \right)
        \right] \right\}
\end{align}

\end{document}
