\section{Gauge Freedom in the Landau Problem}
\label{sec:landau}

\subsection{The Landau Problem}

Next we investigated the degrees of gauge freedom present in the Landau problem
of a uniform everywhere magnetic field, $\mathbf{B} = B \basis{z}$. Of
particular interest is how the ground state wavefunction gains the freedom from
the choice of gauge. This is examined explicitly for the straight (sometimes
called the Landau) and for the circular (sometimes called symmetric) gauges.

\subsection{The Straight Gauge}

In the straight gauge,
\begin{align}
    \mathbf{A} = Bx\basis{y}
    \label{eqn:straight-gauge}
\end{align}
the ground state solution for the wavefunction is
\begin{align}
    \Psi_{s,0}(x,t) = \mathcal{N} e^{- \frac{1}{2} \mymu x^2} e^{-i \frac{1}{2}
        \hbar \Omega t},
\end{align}
where $\mathcal{N}$ is a normalisation factor and $\Omega = \frac{qB}{m}$ is the
angular frequency of the solution \cite{murayama}. Thus the probability density
and phase of the wavefunction are
\begin{align}
    \rho = \mathcal{N}^2 e^{- \mymu x^2} \\
    \mathrm{and}~ \theta = - \frac{1}{2} \hbar \Omega t,
\end{align}
and the probability current is simply
\begin{align}
    \mathbf{J}_s &= \frac{\hbar}{m} \mathcal{N}^2 e^{- \mymu x^2} \left(
        \nabla \left\{ -\frac{1}{2} \hbar \Omega t \right\} - \frac{q}{\hbar} B
        x \basis{y} \right) \\
    &= - \Omega \mathcal{N}^2 e^{- \mymu x^2} x\basis{y} ~\mathrm{as}~
        \nabla \left\{ -\frac{1}{2} \hbar \Omega t \right\} = 0,
\end{align}
as plotted in Figure \ref{fig:straight}

\begin{figure}
    \includegraphics[width=\linewidth]{straight-gauge}
    \caption{The ground state probability current associated with the straight
        gauge, $\mathbf{A} = Bx\basis{y}$.}
    \label{fig:straight}
\end{figure}
