\section{Gauge Invariant Schr\"odinger Equation}
\label{sec:gise}

\subsection{Motivation}

To begin exploring how these unphysical gauge dependent quantities can be
circumvented, we derive a version of the Schr\"odinger equation for a particle
in an electromagnetic field in terms of only gauge invariant quantities. This
will be of particular interest when investigating the Aharanov-Bohm effect (see
FUTURE RESEARCH).

\subsection{The Standard Schr\"odinger Equation}

The Schr\"odinger equation for a particle in an electromagnetic field is given
by:
\begin{align}
    i \hbar \dbydt{\Psi} = \frac{1}{2m}\mathbf{\Pi}^2 \Psi + \phi \Psi
    \label{eqn:schrodinger}
\end{align}
where $\Psi$ is the wavefunction, $m$ is the particle mass, $\phi$ is the
electric potential, $\mathbf{\Pi} = \mathbf{p} - q\mathbf{A}$ is the canonical
momentum, $\mathbf{p} = - i \hbar \nabla$ is the kinetic momentum and
$\mathbf{A}$ is the vector potential \cite[Ch. 4]{aharonov-rohrlich-2008}. This
contains the gauge-dependent and unphysical quantities $\Psi$, $\mathbf{A}$ and
$V$.

Defining the probability density and current as $\rho \defeq \Psi^*\Psi$ and
$\mathbf{J} \defeq \frac{1}{2m}\left( \Psi^* \mathbf{\Pi} \Psi + \Psi
\mathbf{\Pi}^* \Psi^* \right)$, respectively, we can reconstruct Equation
\ref{eqn:schrodinger} in terms of only gauge invariant quantities.

\subsection{Verifying Continuity}

\noindent First we note that
\begin{align}
    \mathbf{J} &= \frac{1}{m} \Ree{\Psi^* \mathbf{\Pi} \Psi} \\
               &= \frac{\hbar}{m} \Imm{\Psi^* \nabla \Psi} - \frac{q}{m} \rho
                    \mathbf{A}
\end{align}
Thus, the divergence of $\mathbf{J}$ is simply:
\begin{align}
    \div{J} &= \frac{\hbar}{m} \Imm{\Psi^* \nabla^2 \Psi} - \frac{q}{m} \left(
                \mathbf{A} \cdot \nabla \rho + \rho \nabla \cdot
                \mathbf{A} \right)
\end{align}
Inputting Equation \ref{eqn:schrodinger} changes causes the terms in
$\mathbf{A}$ to cancel, and converts the \nth{2} order space
derivative to a \nth{1} order time derivative, leaving just:
\begin{align}
    \div{J} &= -2 \Ree{\Psi^* \dbydt{\Psi}} \\
            &= - \left( \Psi^* \dbydt{\Psi}
               + \Psi \dbydt{\Psi^*} \right) \\
            &= - \dbydt{\rho}
\end{align}
Thus we confirm that the continuity equation holds as expected:
\begin{align}
    \div{J} + \dbydt{\rho} = 0
    \label{eqn:continuity}
\end{align}

\subsection{Transfer to Gauge Invariant Quantities}

Using this continuity equation, the definitions for the probability density and
current and the equation for the electric field $\mathbf{E} = - \nabla \phi -
\dbydt{\mathbf{A}}$, we showed that Equation \ref{eqn:schrodinger} can be recast
in terms of only the gauge invariant variables $\mathbf{J}$, $\rho$ and
$\mathbf{E}$.

To see this, we write the wavefunction as:
\begin{align}
    \Psi = \sqrt{\rho} e^{i \theta}
\end{align}
