\section{Gauge Invariant Schr\"odinger Equation}
\label{sec:gise}

To begin exploring how these unphysical gauge dependent quantities can be
circumvented, we derive a version of the Schr\"odinger equation for a particle
in an electromagnetic field in terms of only gauge invariant quantities. This
will be of particular interest when investigating the Aharanov-Bohm effect (see
FUTURE RESEARCH).

The Schr\"odinger equation for a particle in an electromagnetic field is given
by:
\begin{align}
    i \hbar \dbydt{\Psi} = \frac{1}{2m}\mathbf{\Pi}^2 \Psi + V \Psi
    \label{eqn:schrodinger}
\end{align}
where $\Psi$ is the wavefunction, $m$ is the particle mass, $V$ is the electric
potential and $\mathbf{\Pi} = \mathbf{p} - q\mathbf{A}$ is the canonical momentum for
kinetic momentum $\mathbf{p} = - i \hbar \nabla$ and vector potential
$\mathbf{A}$\cite[Ch. 4]{aharonov-rohrlich-2008}. This contains the
gauge-dependent and unphysical quantities $\Psi$, $\mathbf{A}$ and $V$.

Defining the probability density and current as $\rho \defeq \Psi^*\Psi$ and
$\mathbf{J} \defeq \frac{1}{2m}\left( \Psi^* \mathbf{\Pi} \Psi + \Psi
\mathbf{\Pi}^* \Psi^* \right)$, respectively, we can reconstruct Equation
\ref{eqn:schrodinger} in terms of only gauge invariant quantities.

First we note that
\begin{align}
    \mathbf{J} &= \frac{1}{m} \Ree{\Psi^* \mathbf{\Pi} \Psi} \\
               &= \frac{\hbar}{m} \Imm{\Psi^* \nabla \Psi} - \frac{q}{m} \rho
                    \mathbf{A}
\end{align}
Thus, the divergence of $\mathbf{J}$ is simply:
\begin{align}
    \div{J} &= \frac{\hbar}{m} \Imm{\Psi^* \nabla^2 \Psi} - \frac{q}{m} \left(
                \mathbf{A} \cdot \nabla \rho + \rho \nabla \cdot
                \mathbf{A} \right)
\end{align}

Using this, we confirm that the continuity equation holds as expected:
\begin{align}
    \div{J} + \dbydt{\rho} = 0
    \label{eqn:continuity}
\end{align}
