\section{Gauge Invariant Schr\"odinger Equation}
\label{sec:gise}

\subsection{Motivation}

To begin exploring how these unphysical gauge dependent quantities can be
circumvented, we derive a version of the Schr\"odinger equation for a particle
in an electromagnetic field in terms of only gauge invariant quantities. This
will be of particular interest when investigating the Aharanov-Bohm effect (see
FUTURE RESEARCH).

\subsection{The Standard Schr\"odinger Equation}

The Schr\"odinger equation for a particle of mass $m$ and charge $q$ in an
electromagnetic field is given by:
\begin{align}
    i \hbar \dbydt{\Psi} = \frac{1}{2m}\mathbf{\Pi}^2 \Psi + qV \Psi
    \label{eqn:schrodinger}
\end{align}
where $\Psi$ is the wavefunction, $V$ is the electric potential, $\mathbf{\Pi} =
\mathbf{p} - q\mathbf{A}$ is the canonical momentum, $\mathbf{p} = - i \hbar
\nabla$ is the kinetic momentum and $\mathbf{A}$ is the vector potential
\cite[Chapter 4]{aharonov-rohrlich}\cite[Page 456]{landau-lifshitz-quantum}.

This contains the gauge-dependent and unphysical quantities $\Psi$, $\mathbf{A}$
and $V$, but defining the probability density and current as
\begin{align}
    \rho &= \Psi^*\Psi \\
    \mathrm{and}~ \mathbf{J} &= \frac{1}{2m}\left( \Psi^* \mathbf{\Pi}
        \Psi + \Psi \mathbf{\Pi}^* \Psi^* \right),
\end{align}
we can reconstruct Equation \ref{eqn:schrodinger} in terms of only gauge
invariant quantities.

\subsection{Verifying Continuity}

\noindent First we note that
\begin{align}
    \mathbf{J} &= \frac{1}{m} \Ree{\Psi^* \mathbf{\Pi} \Psi} \\
               &= \frac{\hbar}{m} \Imm{\Psi^* \nabla \Psi} - \frac{q}{m} \rho
                    \mathbf{A}
\end{align}
Thus, the divergence of $\mathbf{J}$ is simply:
\begin{align}
    \div{J} &= \frac{\hbar}{m} \Imm{\Psi^* \nabla^2 \Psi} - \frac{q}{m} \left(
                \mathbf{A} \cdot \nabla \rho + \rho \nabla \cdot
                \mathbf{A} \right)
\end{align}
Inputting Equation \ref{eqn:schrodinger} changes causes the terms in
$\mathbf{A}$ to cancel, and converts the \nth{2} order space
derivative to a \nth{1} order time derivative, leaving just:
\begin{align}
    \div{J} &= -2 \Ree{\Psi^* \dbydt{\Psi}} \\
            &= - \left( \Psi^* \dbydt{\Psi}
               + \Psi \dbydt{\Psi^*} \right) \\
            &= - \dbydt{\rho}
\end{align}
Thus we confirm that the continuity equation holds as expected:
\begin{align}
    \div{J} + \dbydt{\rho} = 0
    \label{eqn:continuity}
\end{align}

\subsection{Transfer to Gauge Invariant Quantities}

Using this continuity equation, the definitions for the probability density and
current and the equation for the electric field $\mathbf{E} = - \nabla V -
\dbydt{\mathbf{A}}$, we showed that Equation \ref{eqn:schrodinger} can be recast
in terms of only the gauge invariant variables $\mathbf{J}$, $\rho$ and
$\mathbf{E}$.

To see this, we write the wavefunction as
\begin{align}
    \Psi = \sqrt{\rho} e^{i \theta}, ~ \theta \in \mathbb{R}.
\end{align}
Then the derivatives of $\Psi$ in space and time are
\begin{align}
    \nabla \Psi &= \Psi \left( \frac{1}{2} \frac{\nabla
        \rho}{\rho} + i \nabla \theta \right) \\
    \dot \Psi &= \Psi \left( \frac{1}{2} \frac{\dot
        \rho}{\rho} + i \dot \theta \right),
\end{align}
where $\dot \Psi \defeq \dbydt{\Psi}$. This means that the probability current
can be expressed in the simple form
\begin{align}
    \mathbf{J} = \frac{\hbar}{m} \rho \left( \nabla \theta - \frac{q}{\hbar}
        \mathbf{A} \right)
    \label{eqn:current}
\end{align}
Using this, we show that the time derivative of $\frac{m \mathbf{J}}{\rho}$ is
the simple expression
\begin{align}
    \frac{\partial}{\partial t}\left(\frac{m \mathbf{J}}{\rho}\right)
        = \hbar \nabla \dot \theta - q\mathbf{\dot A}.
\end{align}
Using the Schr\"odinger equation and the continuity equation to replace the
$\dot \theta$ with the spatial derivatives $\nabla^2 \theta$, and replace the
time derivative of $\mathbf{A}$ with $\mathbf{E}$ and $V$ via the equation
\begin{align}
    \mathbf{E} = - \nabla V - \mathbf{\dot A},
\end{align}
we get a third order space derivative in $\rho$, a second order space derivative
in $\theta$ and a first order space derivative in $\mathbf{A}$. Using Equation
\ref{eqn:current}, the terms in $\mathbf{A}$ and $\theta$ can be combined into a
single gauge invariant factor $\nabla \left( \frac{m \mathbf{J}^2}{2 \rho^2}
\right)$, giving a version of the Schr\"odinger equation in terms of only gauge
invariant quantities:
\begin{align}
    \frac{\partial}{\partial t} \left( \frac{m \mathbf{J}}{\rho} \right)
    =& \nabla \left( \frac{\hbar^2}{4m\rho} \nabla^2 \rho -
       \frac{\hbar^2}{8m\rho^2} \left(\nabla \rho \right)^2 - \frac{m
       \mathbf{J}^2}{2 \rho^2} \right) + q\mathbf{E}
    \label{eqn:gise}
\end{align}

Thus, we have a version of the Schr\"odinger equation for a particle in an
electromagnetic field without any reference to gauge dependent quantities like
the wavefunction or potentials. As Aharonov and Rohrlich demonstrate
\cite{aharonov-rohrlich}, this formulation encounters problems when the initial
wavefunction $\Psi(\mathbf{r}, 0)$ is only non-zero in two disjointed regions
(i.e. $\Psi(\mathbf{r}, 0)$ is not simply connected). In particular,
$\rho(\mathbf{r}, 0)$ and $\mathbf{J}(\mathbf{r}, 0)$ do not uniquely determine
the wavefunction $\Psi(\mathbf{r}, t)$ for all future times $t > 0$.
Interestingly, this is the Aharonov-Bohm case; exactly when the vector potential
$\mathbf{A}$ has an important impact of the observable physics is when the gauge
invariant reformulation lacks full predictive power.

In addition, the transfer to gauge invariant variables has increased the order
of the space derivative; where originally the Schr\"odinger equation relates a
first order time derivative to a second order space derivative, Equation
\ref{eqn:gise} relates a first order time derivative to a third order space
derivative. It is clear, then, how much simpler the gauge dependent variables
are to use for actual quantum calculations, particularly when your theory
becomes relativistic as with the Higgs Mechanism
\cite{higgs-1964,englert-brout,guralnik-hagen-kibble}.
% Make sure I don't dig myself into a hole here, I talk earlier about how Higgs
% did indeed formulate his mechanism in gauge invariant terms
