\section{Introduction}
\label{sec:introduction}

The basic aim of this project is to investigate the extent to which the
`unphysicalness' or the gauge dependence of quantum physics can be circumvented
by gauge invariant means. In particular it concerns exploring the possibility
and limits of gauge invariant reformulations of quantum phenomena, and the
degrees of freedom that we are granted in problems where we are forced to choose
a gauge.

We begin with a review of the limited literature concerning the circumvention of
gauge dependent variables, gauge freedoms in the Landau problem and associated
concepts. Secondly, the Schr\"odinger equation, the fundamental equation of
dynamics of quantum mechanics, is reformulated in gauge independent variables
and the scope of this reformulation is discussed. Next, the Landau problem is
investigated and the freedoms associated with the ground state in various
gauges. The symmetries of the ground states are discussed in these gauges, with
particular emphasis on the analytic function that appears in the ground state
circular gauge solution. Specifically, the link between the analyticity of this
function and the continuity of the associated probability current is examined,
and the stringency of the conditions of continuity on the probability current
and analyticity of the introduced degeneracy function is compared.

Next the future directions of research of the project are discussed. Firstly
this includes the chaotic analytic function as a method of illustrating the
difference between the typical divergenceless current and one that is
constrained to correspond to the circular gauge current with the requirement of
the analyticity of the degeneracy function. In addition, the Aharonov-Bohm
effect as an area of future research is discussed. Finally, the gauge
independent density operator and associated Wigner quasiprobability distribution
are mentioned, with reference to the interesting questions that arise concerning
their time evolution and correspondence to classical physics.
