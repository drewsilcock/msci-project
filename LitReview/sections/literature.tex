\section{Literature}
\label{sec:literature}

The literature concerning gauge independent formulations of already established
quantum phenomena is limited, as gauge dependence quantities are significantly
easier to do calculations with, and are almost always the first way that new
phenomena are discovered (if gauge independent formulations are discovered at
all).

An example of this is the Higgs Mechanism. It was first discovered as a theory
using gauge dependent quantities by Higgs; Englert and Brout; and Guralnik,
Hagen and Kibble \cite{higgs-1964,englert-brout,guralnik-hagen-kibble}, but was
later reformulated by Higgs in purely gauge independent terms \cite{higgs-1966}.

Aharonov and Rohrlich explore circumventing gauge dependent variables in
``Quantum Paradoxes'' \cite{aharonov-rohrlich}, where they point out that in the
Aharonov-Bohm case of wavefunction interference over a non-simple connected
region, the reformulation of Schr\"odinger's equation in gauge independent
variables lacks the full predictive power that the standard Schr\"odinger
equation has. This is discussed in more details in the next section.

There is some debate within the philosophy of physics concerning the meaning of
gauges, and whether they are an essential component of the theories or simply
useful for calculations and extensions. In particular Struyve characterises the
debate when he argues that gauge as a concept is merely a ``redundancy in the
state description'', and that every physical law \textit{can} be reformulated in
terms of purely gauge invariant variables \cite{struyve}. Much of the
accompanying philosophical literature concerns whether gauge transformations
refer to nothing more than mapping from solutions to solutions, or whether they
are more fundamentally a consequence of breakdown in determinism
\cite{struyve,earman}. This relates to another contentious concept in said
literature, regarding whether states that differ by some gauge transformation
are genuinely different physical situations, or whether they are simply
different expressions of the same physical system, with merely different values
taken for a redundant mathematical quantity \cite{belot}.

Later the analytic degeneracy function, $f(x,y)$, that occurs in the ground state
wavefunction solution to the Landau problem in the circular gauge is explored.
In particular, the random generation of analytic functions to plot the
``typical'' corresponding probability current is discussed as a method of
illustrating the difference between the ``typical'' divergenceless probability
current and the ``typical'' probability current corresponding to an analytic
$f(x,y)$ in the ground state circular gauge wavefunction. This method of random
analytic function generation, described by Hannay as ``Chaotic Analytic
Function'' generation \cite{hannay}, generates a series of random coefficients
for the Taylor expansion of the analytic function, taken from a Gaussian
distribution with mean chosen to ensure higher order terms can be safely
neglected. Hannay then goes on to describe the calculation of an isotropic
stationary analytic random function from this chaotic analytic function, a
technique which is utilised later.
