\usepackage{mathtools}
\usepackage{amssymb}
\usepackage{graphicx}
\usepackage[super]{nth}
\usepackage{subcaption}
\usepackage{cleveref}
\usepackage{braket}
\usepackage{wrapfig}
\usepackage{color}

\DeclareGraphicsExtensions{.pdf,.eps,.png,.jpg}
\graphicspath{{figures/}}

% New definition of square root:
% it renames \sqrt as \oldsqrt
\let\oldsqrt\sqrt
% it defines the new \sqrt in terms of the old one
\def\sqrt{\mathpalette\DHLhksqrt}
\def\DHLhksqrt#1#2{%
\setbox0=\hbox{$#1\oldsqrt{#2\,}$}\dimen0=\ht0
\advance\dimen0-0.2\ht0
\setbox2=\hbox{\vrule height\ht0 depth -\dimen0}%
{\box0\lower0.4pt\box2}}

\newcommand{\defeq}{\stackrel{\mathclap{\normalfont\mbox{\tiny def}}}{=}}
\newcommand{\dbydt}[1]{\frac{\partial #1}{\partial t}}
\newcommand{\logg}[1]{\log \left\{ #1 \right\}}
\newcommand{\Ree}[1]{\Re \left\{ #1 \right\}}
\newcommand{\Imm}[1]{\Im \left\{ #1 \right\}}
\newcommand{\basis}[1]{\mathbf{\hat #1 }}
\newcommand{\mymu}[0]{\frac{m \Omega}{\hbar}}
\newcommand{\colvec}[3]{\begin{pmatrix} #1 \\ #2 \\ #3 \end{pmatrix}}
\newcommand{\comm}[2]{\left[ #1 , #2 \right]}
\newcommand{\moyal}[2]{\left\{ \left\{ #1, #2 \right\} \right\}}
\newcommand{\poiss}[2]{\left\{ #1, #2 \right\}}

\renewcommand{\div}[1]{\nabla \cdot \mathbf{ #1 }}
