\documentclass{article}

\usepackage{mathtools}
\usepackage{amssymb}
\usepackage{graphicx}
\usepackage[super]{nth}
\usepackage{subcaption}
\usepackage{cleveref}

\DeclareGraphicsExtensions{.pdf,.eps,.png,.jpg}
\graphicspath{{figures/}}

% New definition of square root:
% it renames \sqrt as \oldsqrt
\let\oldsqrt\sqrt
% it defines the new \sqrt in terms of the old one
\def\sqrt{\mathpalette\DHLhksqrt}
\def\DHLhksqrt#1#2{%
\setbox0=\hbox{$#1\oldsqrt{#2\,}$}\dimen0=\ht0
\advance\dimen0-0.2\ht0
\setbox2=\hbox{\vrule height\ht0 depth -\dimen0}%
{\box0\lower0.4pt\box2}}

\newcommand{\defeq}{\stackrel{\mathclap{\normalfont\mbox{\tiny def}}}{=}}
\newcommand{\dbydt}[1]{\frac{\partial #1}{\partial t}}
\newcommand{\logg}[1]{\log \left\{ #1 \right\}}
\newcommand{\Ree}[1]{\Re \left\{ #1 \right\}}
\newcommand{\Imm}[1]{\Im \left\{ #1 \right\}}
\newcommand{\basis}[1]{\mathbf{\hat #1 }}
\newcommand{\mymu}[0]{\frac{m \Omega}{\hbar}}
\newcommand{\colvec}[3]{\begin{pmatrix} #1 \\ #2 \\ #3 \end{pmatrix}}

\renewcommand{\div}[1]{\nabla \cdot \mathbf{ #1 }}


\title{Derivation of Quantum Probability Continuity Equation for EM Schr\"odinger}
\author{Drew Silcock}
\date{}

\begin{document}

\maketitle

Here's the equation we want to derive, called the probability continuity
equation for Quantum Mechanics:

\begin{equation}
    \frac{\partial \rho}{\partial t} + \nabla \cdot \mathbf{J} = 0
\end{equation}

Given the definitions of $\rho$ and $\mathbf{J}$ as follows:

\begin{align}
    \rho &\defeq | \Psi |^2 \\
    \mathbf{J} &\defeq \frac{1}{2m} \Psi^*(-\imath\hbar\nabla -
    \frac{e}{c}\mathbf{A})\Psi +
    \frac{1}{2m}\Psi(\imath\hbar\nabla-\frac{e}{c}\mathbf{A})\Psi^*
\end{align}

Where $\Psi$ is our quantum wavefunction and $\mathbf{A}$ is our vector
potential.

First, let's note that

\begin{align}
    \frac{\partial \rho}{\partial t} &= \frac{\partial}{\partial t} \left(
    \Psi^*\Psi \right) \\
    &= \Psi\frac{\partial \Psi^*}{\partial t} + \Psi^*\frac{\partial
    \Psi}{\partial t}
\end{align}

Before taking the div of $\mathbf{J}$, we first note that

\begin{align}
    \mathbf{J} &= \frac{1}{2m}\left[ \Psi^*\left(-\imath\hbar\nabla\right)\Psi +
    \Psi\left(\imath\hbar\nabla\right)\Psi^*\right] - \frac{e}{mc}\mathbf{A}|\Psi|^2
    \\
    &= \frac{1}{2m}\left[ \left\{ \Psi^*\left(-\imath\hbar\nabla\right)\Psi
    \right\} + \left\{ \Psi^*\left(-\imath\hbar\nabla\right)\Psi \right\}^* \right]
    - \frac{e}{mc}\rho\mathbf{A}
\end{align}

Next we use the fact that $\Re(z) = \frac{z + z^*}{2}$, we see that

\begin{align}
    \mathbf{J} &= -\frac{\hbar}{m}\Re(\Psi^*\imath\nabla\Psi) -
    \frac{e}{mc}\rho\mathbf{A}
\end{align}

Using the equation

\begin{align}
    \Re(\imath z) &= \frac{(\imath z) + (\imath z)^*}{2} \\
    &= \imath\frac{z-z^*}{2} \\
    &= -\frac{z-z^*}{2\imath} \\
    &= -\Im(z)
\end{align}

We can simplify the above to

\begin{equation}
    \mathbf{J} = \frac{\hbar}{m}\Im(\Psi^*\nabla\Psi) -
    \frac{e}{mc}\rho\mathbf{A}
\end{equation}

Now we're ready to take its divergence, and try and show that it's equal to
$\frac{\partial \rho}{\partial t}$:

\begin{align}
    \nabla \cdot \mathbf{J} &= \frac{\hbar}{m} \nabla \cdot \Im(\Psi^* \nabla
    \Psi) - \nabla\left(\frac{e}{mc}\rho\mathbf{A}\right) \\
    &= \frac{\hbar}{m}\Im(\nabla \cdot [\Psi^*\nabla\Psi]) - \frac{e}{mc}\left(
    \nabla \rho \cdot \mathbf{A} + \rho \nabla \cdot \mathbf{A} \right) \\
    &= \frac{\hbar}{m}\Im\left(\nabla \Psi^* \cdot \nabla \Psi + \Psi^*
    \nabla^2\Psi\right) - \frac{e}{mc}\left(\nabla \rho \cdot \mathbf{A} + \rho
    \nabla \cdot \mathbf{A}\right)
\end{align}

We can simply this by realising that

\begin{align}
    \nabla \Psi^* \cdot \nabla \Psi &= \left( \nabla \Psi \right)^* \cdot \nabla
    \Psi \\
    &= |\nabla \Psi|^2 \in \mathbb{R} \\
    \Rightarrow \Im\left(\nabla\Psi^* \cdot \nabla \Psi\right) &= 0
\end{align}

Thus we have

\begin{equation}
    \nabla \cdot \mathbf{J} = \frac{\hbar}{m} \Im \left( \Psi^* \nabla^2
    \Psi\right) - \frac{e}{mc}\left( \nabla \rho \cdot \mathbf{A} + \rho \nabla
    \cdot \mathbf{A} \right)
\end{equation}

Choosing the Coulumb gauge, $\nabla \cdot \mathbf{A} = 0$, allows us to simply
this further to

\begin{equation}
    \nabla \cdot \mathbf{J} = \frac{\hbar}{m}\Im\left(\Psi^*\nabla^2\Psi\right) -
    \frac{e}{mc}\left(\nabla\rho \cdot \mathbf{A}\right)
\end{equation}

Now Schr\"odinger's equation gives us an expression for $\nabla^2\Psi$ which we
might be able to substitute back in

\begin{equation}
    \label{schrodinger}
    \imath\hbar\frac{\partial\Psi}{\partial t} =
    \left[\frac{1}{2m}\left(-\imath\hbar\nabla - \frac{e}{c}\mathbf{A}\right)^2
    + eV\right]\Psi
\end{equation}

But first, let's expand it and rearrange it into a more useful form:

\begin{align}
    \imath\hbar\frac{\partial\Psi}{\partial t} &= \frac{1}{2m}\left(-\hbar^2\nabla^2 +
    \frac{e^2}{c^2}A^2 + \frac{\imath\hbar e}{c}\nabla \cdot \mathbf{A} +
    \frac{\imath\hbar e}{c}\mathbf{A} \cdot \nabla \right)\Psi + eV\Psi \\
    &= \frac{1}{2m}\left(-\hbar^2\nabla^2\Psi + \frac{e^2}{c^2}A^2 +
    \frac{\imath\hbar e}{c}\nabla \cdot [\mathbf{A}\Psi] + \frac{\imath\hbar
    e}{c}\mathbf{A} \cdot [\nabla \Psi] \right) + eV\Psi
\end{align}

By the product rule:

\begin{equation}
    \nabla \cdot (\mathbf{A}\Psi) = \Psi \nabla \cdot \mathbf{A} + \mathbf{A}
    \cdot \nabla \Psi
\end{equation}

Thus, we can write this as:

\begin{equation}
    \imath\hbar\frac{\partial \Psi}{\partial t} = \frac{1}{2m}\left(-\hbar^2\nabla^2\Psi + \frac{e^2}{c^2}A^2 +
    \frac{\imath\hbar e}{c}\Psi \nabla \cdot \mathbf{A} + \frac{2\imath\hbar
    e}{c}\mathbf{A} \cdot \nabla \Psi \right) + eV\Psi
\end{equation}

Remembering that $\nabla \cdot \mathbf{A} = 0$, due to our choice of gauge:

\begin{align}
    \imath\hbar\frac{\partial\Psi}{\partial t} &=
    \frac{1}{2m}\left(-\hbar^2\nabla^2 + \frac{e^2}{c^2}A^2 + \frac{2\imath\hbar
    e}{c}\mathbf{A} \cdot \nabla \right) \Psi + eV\Psi \\
    -\frac{\hbar^2}{2m}\nabla^2\Psi &= \imath\hbar\frac{\partial\Psi}{\partial
    t} - \frac{e^2}{c^2}A^2\Psi - \frac{\imath\hbar e}{mc}\mathbf{A} \cdot \nabla
    \Psi + eV\Psi \\
    \nabla^2\Psi &= -\frac{2m\imath}{\hbar}\frac{\partial\Psi}{\partial t} +
    \frac{2me^2}{\hbar^2c^2}A^2\Psi + \frac{2\imath e}{\hbar c}\mathbf{A} \cdot
    \nabla\Psi + eV\Psi
\end{align}

As $V$ is a real scalar, $\Psi^* eV\Psi = eV |\Psi^2| \in \mathbb{R}$, thus
this term goes to zero when we take the imaginary part. Our expression thus
becomes:

\begin{align}
    \nabla \cdot \mathbf{J} &= \frac{\hbar}{m}
    \Im\left(\Psi^*\left[-\frac{2m\imath}{\hbar}\frac{\partial\Psi}{\partial t}
    + \frac{2me^2}{\hbar^2 c^2}A^2 \Psi + \frac{2\imath e}{\hbar c}\mathbf{A}
    \cdot \nabla \Psi \right]\right) \notag \\
    & \hspace{12pt} - \frac{e}{mc}\left(\nabla \rho \cdot
    \mathbf{A}\right) \\
\end{align}

Assuming $\mathbf{A} \in \mathbb{R}$, the term
$\Im\left(\Psi^*\frac{2me^2}{\hbar^2c^2}A^2\Psi\right) = 0$, meaning we have the
following:

\begin{equation}
    \nabla \cdot \mathbf{J} =
    \frac{\hbar}{m}\Im\left(-\frac{2m\imath}{\hbar}\Psi^*\frac{\partial\Psi}{\partial
    t} + \frac{2\imath e}{\hbar c}\Psi^* \mathbf{A} \cdot \nabla \Psi\right) -
    \frac{e}{mc}\left(\nabla\rho \cdot \mathbf{A}\right)
    \label{eqn:penultimate}
\end{equation}

Let's look at that second term in the $\Im$:

\begin{align}
    \frac{\hbar}{m}\Im\left(\frac{2\imath e}{\hbar c} \Psi^* \mathbf{A} \cdot
    \nabla \Psi\right) =&
    \frac{2e}{mc}\Im\left(\imath\mathbf{A}\cdot[\Psi^*\nabla\Psi]\right) \\
    & \frac{2e}{mc}\Re\left(\mathbf{A}\cdot[\Psi^*\nabla\Psi]\right) \\
    & \frac{e}{mc}\left(\left\{\mathbf{A} \cdot [\Psi^*\nabla\Psi]\right\} +
    \left\{\mathbf{A} \cdot [\Psi^*\nabla\Psi]\right\}^*\right) \\
    & \frac{e}{mc}\left( \mathbf{A} \cdot [ \Psi^*\nabla\Psi + \Psi\nabla\Psi^*
    ] \right)
\end{align}

Remembering that $\nabla\rho = \nabla(\Psi^*\Psi) = \Psi^*\nabla\Psi + \Psi\nabla\Psi^*$,
this can be simplified to:

\begin{align}
    \frac{e}{mc}(\mathbf{A}\cdot[\Psi^*\nabla\Psi + \Psi\nabla\Psi^*]) &=
    \frac{e}{mc}(\mathbf{A}\cdot\nabla\rho) \\
    &= \frac{e}{mc}(\nabla\rho\cdot\mathbf{A})
\end{align}

Inputting this back into Equation \ref{eqn:penultimate} gives:

\begin{align}
    \nabla \cdot \mathbf{J} &= \frac{\hbar}{m} \Im
    \left(-\frac{2m\imath}{\hbar}\Psi^* \frac{\partial \Psi}{\partial t}\right)
     + \frac{e}{mc}(\nabla \rho \cdot \mathbf{A}) - \frac{e}{mc}(\nabla \rho
    \cdot \mathbf{A}) \\
    &= \frac{\hbar}{m} \Im \left(-\frac{2m\imath}{\hbar}\Psi^* \frac{\partial
    \Psi}{\partial t}\right) \\
    &= - \Re \left( 2 \Psi^* \frac{\partial \Psi}{\partial t} \right) \\
    &= - \left( \left\{ \Psi^* \frac{\partial \Psi}{\partial t} \right\} +
    \left\{ \Psi^* \frac{\partial \Psi}{\partial t} \right\}^* \right) \\
    &= - \left( \Psi^* \frac{ \partial \Psi }{ \partial t } + \Psi \frac{
    \Psi^*}{\partial t} \right)
\end{align}

We showed at the beginning in Equation \label{eqn:density-change} that this is
simply:

\begin{align}
    \nabla \cdot \mathbf{J} &= - \frac{\partial \rho}{\partial t} \\
    \Rightarrow \frac{\partial \rho}{\partial t} + \nabla \cdot \mathbf{J}
    &= 0
\end{align}

We have thus proven the continuity equation demonstrating conservation of
probability in quantum mechanics, from the definitions of $\rho$ and
$\mathbf{J}$.

$\square$

\end{document}
