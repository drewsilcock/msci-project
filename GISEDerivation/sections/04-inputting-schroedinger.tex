%----------------------------------------------------------------------------%
%-------------------- Inputting Schroedinger equation------------------------%
%----------------------------------------------------------------------------%

\section{Inputting the Schr\"odinger equation}
\label{sec:inputting}

We now recognise that $\dot \theta$ can be given by the Schr\"odinger equation:

\begin{equation}
    \imath \hbar \dot \Psi = \frac{1}{2m} \left(-\imath \hbar \nabla - \frac{e}{c}
    \mathbf{A} \right)^2 \Psi + e \phi \Psi
    \label{eqn:schroedinger}
\end{equation}

But first we need to rearrange the Schr\"odinger equation into a form more
useful to us, involving $\theta$ and $\rho$ instead of $\Psi$.

To do this, we can start by expanding the round brackets:

\begin{align}
    \left( - \imath \hbar \nabla - \frac{e}{c} \mathbf{A} \right)^2 \Psi &=
    \left( - \imath \hbar \nabla \right)^2 \Psi + \frac{e^2}{c^2} A^2 \Psi \\
    & \hspace{11.5pt} + \left( \imath \hbar \nabla \right) \cdot \left( \frac{e}{c} \mathbf{A}
    \right) \Psi + \left( \frac{e}{c} \mathbf{A} \right) \cdot \left( \imath \hbar \nabla
    \right)
    \Psi \\
    &= - \hbar^2 \nabla^2 \Psi + \frac{e^2}{c^2} A^2 \Psi + \frac{ \imath \hbar
    e}{c} \nabla \cdot \left( \mathbf{A} \Psi \right) + \frac{ \imath \hbar e}{c}
    \mathbf{A} \cdot \nabla \Psi \\
    &= - \hbar^2 \nabla^2 \Psi + \frac{e^2}{c^2} A^2 \Psi + \frac{ \imath \hbar
    e}{c} \left( \nabla \cdot \mathbf{A} \right) \Psi + \frac{ 2\imath \hbar e}{c}
    \mathbf{A} \cdot \nabla \Psi
\end{align}

By choice of gauge, $\nabla \cdot \mathbf{A} = 0$, so:

\begin{equation}
    \left( - \imath \hbar \nabla - \frac{e}{c} \mathbf{A} \right)^2 \Psi = -
    \hbar^2 \nabla^2 \Psi + \frac{e^2}{c^2} A^2 \Psi + \frac{2 \imath \hbar
    e}{c} \mathbf{A} \cdot \nabla \Psi
\end{equation}

%----------------------------------------------------------------------------%
%---------------------------- Right Hand Side -------------------------------%
%----------------------------------------------------------------------------%

\subsection{Right Hand Side (RHS)}
\label{sec:schroedinger/rhs}

So now we can start replacing $\Psi$ with $\sqrt{\rho} e^{\imath \theta}$ in
parts:

\begin{align}
    \nabla \Psi &= \nabla \left( \sqrt{\rho} e^{\imath \theta} \right) \\
                &= \nabla \left( \sqrt{\rho} \right) e^{\imath \theta} +
    \sqrt{\rho} \nabla \left( e^{\imath \theta} \right) \\
    &= \frac{\nabla \rho}{2 \sqrt{\rho}} e^{\imath \theta} + \imath \nabla \theta
    \sqrt{\rho} e^{\imath \theta} \\
    &= \Psi \left( \frac{\nabla \rho}{2 \rho} + \imath \nabla \theta
    \right) \label{eqn:nabla-psi}
\end{align}

This means that:

\begin{align}
    \nabla^2 \Psi &= \nabla \left[ \Psi \left( \frac{\nabla \rho}{2 \rho} +
    \imath \nabla \theta \right) \right] \\
    &= \nabla \Psi \left( \frac{\nabla \rho}{2 \rho} + \imath \nabla \theta
        \right) + \Psi \left( \nabla \left[ \frac{\nabla \rho}{2 \rho} \right] +
        \imath \nabla^2 \theta \right) \\
    &= \nabla \Psi \left( \frac{\nabla \rho}{2 \rho} + \imath \nabla \theta
        \right) + \Psi \left( \frac{\nabla^2 \rho}{2 \rho} + \frac{\nabla
        \rho}{2} \nabla \left( \frac{1}{\rho} \right) +
        \imath \nabla^2 \theta \right) \\
    &= \nabla \Psi \left( \frac{\nabla \rho}{2 \rho} + \imath \nabla \theta
        \right) + \Psi \left( \frac{\nabla^2 \rho}{2 \rho} -
            \frac{ \left[ \nabla \rho \right]^2}{2\rho^2} +
        \imath \nabla^2 \theta \right) \label{eqn:nabla-squared-psi}
\end{align}

Using Equation \ref{eqn:nabla-psi}, we can expand this first term:

\begin{align}
    \nabla \Psi \left( \frac{ \nabla \rho}{2 \rho} + \imath \nabla \theta
    \right)
    &= \Psi \left( \frac{\nabla \rho}{2\rho} + \imath \nabla \theta \right)^2 \\
    &= \Psi \left( \frac{\left[ \nabla \rho \right]^2}{4 \rho^2} - \left[ \nabla \theta
        \right] ^2 + \imath \frac{\nabla \rho \cdot \nabla \theta}{\rho} \right)
        \label{eqn:nabla-psi-bracket}
\end{align}

Thus, inputting Equation \ref{eqn:nabla-psi-bracket} into the equation for
$\nabla^2 \Psi$ gives:

\begin{align}
    \nabla^2 \Psi
    &= \Psi \left[ \frac{\nabla^2 \rho}{2 \rho} - \frac{ \left(
        \nabla \rho \right)^2}{4 \rho^2} - \left( \nabla \theta \right)^2 + \imath
        \nabla^2 \theta + \imath \frac{\nabla \rho \cdot \nabla \theta}{\rho} \right]
\end{align}

Now that we have both $\nabla \Psi$ and $\nabla^2 \Psi$, we know that the round
bracket squared term in the Schr\"odinger equation is:

\begin{align}
    \left( -\imath \hbar \nabla - \frac{e}{c} \mathbf{A} \right)^2 \Psi
    &= -\hbar^2 \Psi \left[ \frac{\nabla^2 \rho}{2 \rho} - \frac{ \left(
        \nabla \rho \right)^2}{4 \rho^2} - \left( \nabla \theta \right)^2 + \imath
        \nabla^2 \theta + \imath \frac{\nabla \rho \cdot \nabla \theta}{\rho} \right]
        \notag \\
      & \hspace{11.5pt} + \frac{e^2}{c^2} A^2 \Psi + \frac{2 \imath \hbar e}{c}
        \mathbf{A} \cdot \left( \frac{\nabla \rho}{2 \rho} + \imath \nabla
        \theta \right) \Psi \\
    &=  \Psi \Bigg[ - \hbar^2 \frac{\nabla^2 \rho}{2\rho} + \hbar^2 \frac{\left(
        \nabla \rho \right)^2}{4 \rho^2} + \hbar^2 \left( \nabla \theta
        \right)^2 - \imath \hbar^2
        \nabla^2 \theta
        \notag \\
      & \hspace{30pt} - \imath \hbar^2 \frac{\nabla \rho \cdot \nabla
        \theta}{\rho} + \frac{e^2}{c^2} A^2 + \frac{\imath \hbar e}{c}
        \mathbf{A} \cdot \frac{\nabla \rho}{\rho} \\
      & \hspace{30pt} - \frac{2 \hbar e}{c} \mathbf{A} \cdot \nabla \theta
        \Bigg]
\end{align}

Then the full RHS of the Schr\"odinger equation is given by the monolithic
expression:

\begin{align}
    \mathrm{RHS}
    &=  \Psi \Bigg[ - \frac{\hbar^2}{2m} \frac{\nabla^2 \rho}{2\rho} +
        \frac{\hbar^2}{2m} \frac{\left( \nabla \rho \right)^2}{4 \rho^2} +
        \frac{\hbar^2}{2m} \left( \nabla \theta \right)^2 - \imath
        \frac{\hbar^2}{2m} \nabla^2 \theta
        \notag \\
        & \hspace{30pt} - \imath \frac{\hbar^2}{2m} \frac{\nabla \rho \cdot \nabla
    \theta}{\rho} + \frac{e^2}{2mc^2} A^2 + \frac{\imath \hbar e}{2mc}
        \mathbf{A} \cdot \frac{\nabla \rho}{\rho} \\
      & \hspace{30pt} - \frac{\hbar e}{mc} \mathbf{A} \cdot \nabla \theta + e
        \phi \Bigg] \notag
\end{align}

%----------------------------------------------------------------------------%
%--------------------------- Left Hand Side ---------------------------------%
%----------------------------------------------------------------------------%

\subsection{Left Hand Side (LHS)}
\label{sec:schroedinger/lhs}

Now that we've sufficiently expanded out the RHS of the Schr\"odinger equation,
let's look at the left hand side. By the same logic that $\nabla \Psi = \Psi
\left( \frac{\nabla \rho}{2\rho} + \imath \nabla \theta \right)$:

\begin{equation}
    \dot \Psi = \Psi \left( \frac{\dot \rho}{2\rho} + \imath \dot \theta \right)
    \label{eqn:psi-dot}
\end{equation}

Inputting this into the LHS of Schr\"odinger gives:

\begin{equation}
    \imath \hbar \dot \Psi = \imath \hbar \Psi \left( \frac{\dot \rho}{2\rho} +
    \imath \dot \theta \right)
\end{equation}

We can eliminate the $\dot \rho$ using the continuity equation:

\begin{equation}
    \dot \rho + \nabla \cdot \mathbf{J} = 0
    \label{eqn:continuity}
\end{equation}

This gives for the LHS:

\begin{align}
    \imath \hbar \dot \Psi &= \imath \hbar \Psi \left( - \frac{\nabla \cdot
    \mathbf{J}}{2\rho} + \imath \dot \theta \right)
\end{align}

We can simply the divergence of $\mathbf{J}$ using Equation \ref{eqn:j}:

\begin{align}
    \nabla \cdot \mathbf{J}
    &= \nabla \cdot \left[ \frac{\hbar}{m} \rho \left( \nabla \theta -
        \frac{e}{\hbar c} \mathbf{A} \right) \right] \\
    &= \frac{\hbar}{m} \left[ \nabla \rho \cdot \left( \nabla \theta -
        \frac{e}{\hbar c} \mathbf{A} \right) + \rho \left( \nabla^2 \theta -
        \frac{e}{\hbar c} \nabla \cdot \mathbf{A} \right) \right]
\end{align}

Again by the choice of the Coulomb gauge, $\nabla \cdot \mathbf{A} = 0$, so:

\begin{align}
    \nabla \cdot \mathbf{J}
    &= \frac{\hbar}{m} \left[ \nabla \rho \cdot \nabla \theta - \frac{e}{\hbar c}
        \mathbf{A} \cdot \nabla \rho + \rho \nabla^2 \theta \right] \\
    &= \frac{\hbar}{m} \nabla \rho \cdot \nabla \theta - \frac{e}{mc}
        \mathbf{A} \cdot \nabla \rho + \frac{\hbar}{m} \rho \nabla^2 \theta
\end{align}

So the LHS of the Schr\"odinger equation in full is:

\begin{align}
    \mathrm{LHS}
    &= \imath \hbar \dot \Psi \\
    &= \imath \hbar \Psi \left[ - \frac{\hbar}{m} \frac{\nabla \rho \cdot
        \nabla \theta}{2\rho} + \frac{e}{mc} \mathbf{A} \cdot \frac{\nabla
        \rho}{2\rho} - \frac{\hbar}{2m} \nabla^2 \theta + \imath \dot \theta
        \right] \\
    &= \Psi \left[ - \imath \frac{\hbar^2}{2m} \frac{\nabla \rho \cdot
        \nabla \theta}{\rho} + \imath \frac{\hbar e}{2mc} \mathbf{A} \cdot
        \frac{\nabla \rho}{\rho} - \imath \frac{\hbar^2}{2m} \nabla^2 \theta -
        \hbar \dot \theta \right]
\end{align}

%----------------------------------------------------------------------------%
%----------------------- Setting LHS and RHS equal --------------------------%
%----------------------------------------------------------------------------%

\newcommand{\lhs}[0]{\Psi \left[ - \imath \frac{\hbar^2}{2m} \frac{\nabla \rho
        \cdot \nabla \theta}{\rho} + \imath \frac{\hbar e}{2mc} \mathbf{A} \cdot
        \frac{\nabla \rho}{\rho} - \imath \frac{\hbar^2}{2m} \nabla^2 \theta -
        \hbar \dot \theta \right]}

\newcommand{\rhs}[0]{\Psi \Bigg[ - \frac{\hbar^2}{2m} \frac{\nabla^2
        \rho}{2\rho} + \frac{\hbar^2}{2m} \frac{\left( \nabla \rho \right)^2}{4
        \rho^2} + \frac{\hbar^2}{2m} \left( \nabla \theta \right)^2 - \imath
        \frac{\hbar^2}{2m} \nabla^2 \theta \notag \\ & \hspace{30pt} - \imath
        \frac{\hbar^2}{2m} \frac{\nabla \rho \cdot \nabla \theta}{\rho} +
        \frac{e^2}{2mc^2} A^2 + \imath \frac{\hbar e}{2mc} \mathbf{A} \cdot
        \frac{\nabla \rho}{\rho} \\ & \hspace{30pt} - \frac{\hbar e}{mc}
        \mathbf{A} \cdot \nabla \theta + e \phi \Bigg] \notag}

\subsection{Setting LHS and RHS equal to each other}

As the title of this subsection suggests, what we do next is, not illogically,
set both the LHS and the RHS of the rearranged and monolithic Schr\"odinger
equation equal to each other. Here it is, but I'll warn you in advance that it's
not pretty:

\begin{align}
    \mathrm{LHS} &= \lhs \\
  = \mathrm{RHS} &= \rhs
\end{align}

Immediately we can cancel the $\Psi$ on both sides, as well as the $- \imath
\frac{\hbar^2}{2m} \frac{\nabla \rho \cdot \nabla \theta}{\rho}$ terms, the
$\imath \frac{\hbar e}{2mc} \mathbf{A} \cdot \frac{\nabla \rho}{\rho}$ terms and
the $- \imath \frac{\hbar^2}{2m} \nabla^2 \theta$ terms. This cuts the equation
down somewhat to:

\begin{align}
    - \hbar \dot \theta
    &=  - \frac{\hbar^2}{4m} \frac{\nabla^2 \rho}{\rho}
        + \frac{\hbar^2}{8m} \frac{ \left( \nabla \rho \right)^2}{\rho^2}
        + \frac{\hbar^2}{2m} \left( \nabla \theta \right)^2
        + \frac{e^2}{2mc^2}A^2 \notag \\
    & \hspace{11.5pt} - \frac{\hbar e}{mc} \mathbf{A} \cdot \nabla \theta
        + e \phi \\
    \hbar \dot \theta + e \phi
    &=  \frac{\hbar^2}{4m} \frac{\nabla^2 \rho}{\rho}
        - \frac{\hbar^2}{8m} \frac{ \left( \nabla \rho \right)^2}{\rho^2}
        - \frac{\hbar^2}{2m} \left( \nabla \theta \right)^2
        - \frac{e^2}{2mc^2}A^2
        + \frac{\hbar e}{mc} \mathbf{A} \cdot \nabla \theta
\end{align}

Taking the gradient of both sides of this equation gets us back to something
that should look familiar from Section \ref{sec:electric}:

\begin{align}
    \hbar \nabla \dot \theta + e \nabla \phi
    = \nabla \bigg[&
      \frac{\hbar^2}{4m} \frac{\nabla^2 \rho}{\rho}
    - \frac{\hbar^2}{8m} \frac{ \left( \nabla \rho \right)^2}{\rho^2}
    - \frac{\hbar^2}{2m} \left( \nabla \theta \right)^2
    \notag \\ &
    - \frac{e^2}{2mc^2} A^2
    + \frac{\hbar e}{mc} \mathbf{A} \cdot \nabla \theta \bigg]
\end{align}

We now substitute the left hand side of this into Equation \ref{eqn:mj-over-rho}
to get:

\begin{align}
    \frac{\partial}{\partial t} \left( \frac{m \mathbf{J}}{\rho} \right)
    &=  \hbar \nabla \dot \theta + e \nabla \phi + e \mathbf{E} \\
    &=  \nabla \bigg[
      \frac{\hbar^2}{4m} \frac{\nabla^2 \rho}{\rho}
    - \frac{\hbar^2}{8m} \frac{ \left( \nabla \rho \right)^2}{\rho^2}
    - \frac{\hbar^2}{2m} \left( \nabla \theta \right)^2
    \notag \\ & \hspace{28pt}
    - \frac{e^2}{2mc^2} A^2
    + \frac{\hbar e}{mc} \mathbf{A} \cdot \nabla \theta \bigg]
    + e \mathbf{E}
    \label{eqn:end-of-sec-4}
\end{align}

This is very close to the final gauge invariant form that we want, but we've
still got the gauge variant $\mathbf{A}$ in there.
