%----------------------------------------------------------------------------%
%-------------------------------- Conclusion --------------------------------%
%----------------------------------------------------------------------------%

\section{Conclusion}
\label{sec:conclusion}

In conclusion, we've used but four key facts:

\begin{description}
    \item[Definition of phase:]
        $$ \Psi = \sqrt{\rho} e^{\imath \theta} $$
    \item[Definition of vector potential:]
        $$ \mathbf{E} = - \frac{1}{c} \mathbf{\dot A} - \nabla \phi $$
    \item[Continuity equation:]
        $$ \dot \rho + \nabla \cdot \mathbf{J} = 0 $$
    \item[Schr\"odinger's equation:]
        $$ \imath \hbar \dot \Psi = \frac{1}{2m} \left( - \imath \hbar \nabla -
           \frac{e}{c} \mathbf{A} \right)^2 \Psi + e \phi \Psi $$
\end{description}

Using these, we've gone from the standard Schr\"odinger equation, which involves
gauge dependent terms $\Psi$ and $\mathbf{A}$, to a gauge invariant form,
involving only the probability density $\rho$, the probability density
$\mathbf{J}$ and the electric field, $\mathbf{E}$, all of which are
gauge-independent observable quantities.

\vfill

\textbf{Note:} The above calculations were all performed using CGS units. To do
this in SI is a simple matter of replacing all $\frac{e}{c} \mathbf{A}$ with $e
\mathbf{A}$. It doesn't actually matter in terms of the final answer you get,
because all the factors of $\mathbf{A}$ cancel.
