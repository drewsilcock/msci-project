\documentclass{article}

\usepackage{mathtools}
\usepackage{amssymb}
\usepackage{graphicx}
\usepackage[super]{nth}
\usepackage{subcaption}
\usepackage{cleveref}

\DeclareGraphicsExtensions{.pdf,.eps,.png,.jpg}
\graphicspath{{figures/}}

% New definition of square root:
% it renames \sqrt as \oldsqrt
\let\oldsqrt\sqrt
% it defines the new \sqrt in terms of the old one
\def\sqrt{\mathpalette\DHLhksqrt}
\def\DHLhksqrt#1#2{%
\setbox0=\hbox{$#1\oldsqrt{#2\,}$}\dimen0=\ht0
\advance\dimen0-0.2\ht0
\setbox2=\hbox{\vrule height\ht0 depth -\dimen0}%
{\box0\lower0.4pt\box2}}

\newcommand{\defeq}{\stackrel{\mathclap{\normalfont\mbox{\tiny def}}}{=}}
\newcommand{\dbydt}[1]{\frac{\partial #1}{\partial t}}
\newcommand{\logg}[1]{\log \left\{ #1 \right\}}
\newcommand{\Ree}[1]{\Re \left\{ #1 \right\}}
\newcommand{\Imm}[1]{\Im \left\{ #1 \right\}}
\newcommand{\basis}[1]{\mathbf{\hat #1 }}
\newcommand{\mymu}[0]{\frac{m \Omega}{\hbar}}
\newcommand{\colvec}[3]{\begin{pmatrix} #1 \\ #2 \\ #3 \end{pmatrix}}

\renewcommand{\div}[1]{\nabla \cdot \mathbf{ #1 }}


\title{The Probability Current for the Landau Problem}
\author{Drew Silcock}
\longdate{}

\begin{document}

\maketitle

\section{The Straight/Landau Gauge}

The Landau problem is trying to find the wavefunction for a uniform everywhere
magnetic field:
\begin{equation}
    \mathbf{B} = B \basis{z}
\end{equation}

We will only consider the ground state solutions here. In order to find a
solution for the wavefunction $\Psi$, we must first choose a gauge. For the
straight gauge, $\mathbf{A} = B x \basis{y}$, the ground state solution is:
\begin{align}
    \Psi_{s,0}(x,t) = \mathcal{N} e^{- \frac{1}{2} \frac{M \Omega}{\hbar} x^2} e^{-i
    \frac{1}{2} \hbar \Omega t}
\end{align}
Where $\mathcal{N}$ is a normalisation factor, $M$ is the mass of the particle,
$\Omega \defeq \frac{q B}{M} \in \mathbb{R}$ is the angular frequency.
% TODO: Clearer explanation of \Omega

Writing $\Psi$ as $\Psi = \sqrt{\rho} e^{i \theta}$ where $\rho$ is the
probability density and $\theta$ is the phase of the wavefunction, we find:
\begin{align}
    \rho &= \mathcal{N}^2 e^{- \frac{M \Omega}{\hbar} x^2} \\
    \textrm{and}~ \theta &= - \frac{1}{2} \hbar \Omega t
\end{align}

Thus, the probability current for the straight gauge is:
\begin{align}
    \mathbf{J}_s &= \frac{\hbar \rho}{M} \left( \nabla \theta - \frac{q}{\hbar}
        \mathbf{A} \right) \\
    &= \frac{\hbar}{M} \mathcal{N}^2 e^{- \frac{M \Omega}{\hbar} x^2}
       \left( \nabla \left\{
       - \frac{1}{2} \hbar \Omega t \right\}
       - \frac{q}{\hbar} B x \basis{y} \right) \\
    &= - \frac{q B}{M} \mathcal{N}^2 e^{- \frac{M \Omega}{\hbar} x^2} x \basis{y}
       ~\textrm{as}~ \nabla \left\{ - \frac{1}{2} \hbar \Omega t \right\} = 0 \\
    &= - \mathcal{N}^2 \Omega e^{- \frac{M \Omega}{\hbar} x^2} x \basis{y}
\end{align}

\section{The Circular Gauge}

If we take the circular gauge, $\mathbf{A} = \frac{1}{2} B \left( x \basis{y} -
y \basis{x} \right)$, then we find the ground state wavefunction is:
\begin{align}
    \Psi_{c,0}(x,y,t) = \mathcal{N} f(x,y)
        e^{- \frac{1}{4}\frac{M\Omega}{\hbar} \left( x^2 + y^2 \right)}
        e^{- i \frac{1}{2} \hbar \Omega t}
\end{align}
where $f(x,y)$ is an complex analytic function of $x$ and $y$ and $\mathcal{N}$
is a different, generic normalisation.

If we write this analytic function $f(x,y)$ as $f(x,y) = F e^{i \phi}$, then the
associated probability current is:
\begin{align}
    \mathbf{J}_c(F, \phi) &= \frac{\hbar \rho}{M} \left( \nabla \theta - \frac{q}{\hbar}
        \mathbf{A} \right) \\
    &= \frac{\hbar F^2}{M} \mathcal{N}^2 e^{- \frac{1}{2} \frac{M \Omega}{\hbar} \left( x^2 + y^2
       \right)} \left( \nabla \phi - \frac{q}{\hbar} \frac{1}{2} B \left( x\basis{y} -
       y\basis{x} \right) \right) \\
       &= \frac{\hbar F^2}{M} \mathcal{N}^2 e^{- \frac{1}{2} \frac{M \Omega}{\hbar} \left( x^2 + y^2
       \right)} \left( \nabla \phi - \frac{M \Omega}{2 \hbar} \left( x\basis{y} -
       y\basis{x} \right) \right)
\end{align}

\section{Equivalence of straight and circular}

We can choose this analytic function $f(x,y)$ such that the probability current
for the circular gauge reduces to the probability current for the straight
gauge. If we choose:
\begin{align}
    f_s(x,y) &= e^{- \frac{1}{4} \frac{M \Omega}{\hbar} \left( x + i y \right)^2}
    \\
    &= e^{- \frac{1}{4} \frac{M \Omega}{\hbar} \left( x^2 - y^2 \right)}
        e^{- i \frac{1}{2} \frac{M \Omega}{\hbar} xy} \\
        \Rightarrow F_s &= e^{- \frac{1}{4} \frac{M \Omega}{\hbar} \left( x^2 - y^2
    \right)} \\
    \textrm{and}~ \phi_s &= -\frac{1}{2} \frac{M \Omega}{\hbar} xy
\end{align}

Then $\mathbf{J}_c$ becomes:
\begin{align}
    \mathbf{J}_c(F_s, \phi_s)
    &= \frac{\hbar F_s^2}{M} \mathcal{N}^2 e^{- \frac{1}{2} \frac{M \Omega}{\hbar} \left( x^2 + y^2
       \right)} \left( \nabla \phi_s - \frac{M \Omega}{2 \hbar} \left( x\basis{y} -
       y\basis{x} \right) \right) \\
    &= \frac{\hbar e^{- \frac{1}{2} \frac{M \Omega}{\hbar} \left( x^2 - y^2
       \right)}}{M} \mathcal{N}^2 e^{- \frac{1}{2} \frac{M \Omega}{\hbar} \left( x^2
       + y^2 \right)} \left( \nabla \left\{ - \frac{1}{2} \frac{M \Omega}{\hbar} xy
       \right\} - \frac{M \Omega}{2 \hbar} \left( x\basis{y} - y \basis{x}
       \right) \right) \\
    &= -\frac{\Omega}{2} \mathcal{N}^2 e^{- \frac{1}{2} \frac{M
       \Omega}{\hbar} x^2} \left( \nabla \left\{ xy \right\} + x\basis{y} -
       y\basis{x} \right) \\
    &= -\frac{\Omega}{2} \mathcal{N}^2 e^{- \frac{1}{2} \frac{M
       \Omega}{\hbar} x^2} \left( x\basis{y} + y\basis{x} + x\basis{y} -
       y\basis{x} \right) \\
    &= -\mathcal{N}^2 \Omega e^{- \frac{1}{2} \frac{M \Omega}{\hbar} x^2}
       x\basis{y} \\
       &= \mathbf{J}_s
\end{align}

\end{document}
